\section{Conclusioni}
I modelli addestrati hanno tutti ottime performance. La scelta del modello migliore,
che verrà effettivamente usato per la classificazione dei fagioli,
ricade, quindi, principalmente sul tempo di addestramento.
La rete neurale, offrendo prestazioni identiche alla SVM ma con un tempo 
di addestramento maggiore, è il primo modello che è possibile scartare.
Lo stesso ragionamento può essere applicato all'albero decisionale, che offre
prestazioni identiche al Gaussian Naive Bayes classifier ma con un tempo di addestramento
16 volte maggiore.
Tra i due modelli rimanenti, ovvero SVM e Gaussian Naive Bayes, la scelta dipende
dall'accuratezza (bilanciata) richiesta dal dominio: se, infatti, essa deve essere
massimizzata, allora la scelta ricade sulla SVM, e saranno tollerati
tempi di addestramento maggiori.
Se invece il dominio di applicazione tollera permormance minori (che in questo
caso sono comunque ottime, in quanto Gaussian Naive Bayes ha un'accuratezza bilanciata
media del $91,5\%$), allora è possibile scegliere Gaussian Naive Bayes,
e beneficiare di tempi di addestramento molto ridotti.

Nel dominio di applicazione considerato in questo progetto, ovvero la classificazione
di fagioli nelle loro varietà, alcuni errori possono essere tollerati, 
in quanto campo a basso impatto d'errore. Di conseguenza, il modello che
sarà mandato in produzione sarà il Gaussian Naive Bayes.

Questo modello permetterà di diminuire il tempo richiesto per la classificazione,
e i costi saranno notevolmente ridotti, in quanto non sarà più necessario assumere
personale addetto (o comunque farlo in maniera molto ridotta).
Il principale costo associato al sistema di riconoscimento automatico sarà
solo l'investimento iniziale per l'acquisto del sistema stesso.