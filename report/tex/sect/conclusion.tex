\section{Conclusioni}
Tutti i modelli scelti ottengono ottime performance sul dataset: si riscontrano
sempre, infatti, macro-precisione, macro-recall e macro-f1 score maggiori del 90\%.
Come già affermato, però, i tempi di addestramento variano notevolmente,
e la scelta del migliore modello deve essere dettata anche da questo fattore.
Dopo un'attenta analisi, si è deciso che il modello più adatto al dominio considerato
è il Gaussian Naive Bayes, con un macro-f1 score medio del $91,5\%$
e un tempo di addestramento quasi istantaneo, che si aggira intorno ai 6 ms. 

Questo modello permetterà di diminuire il tempo richiesto per la classificazione,
dei fagioli, e i costi saranno notevolmente ridotti, in quanto non sarà più 
necessario assumere personale addetto (o comunque lo si farà in maniera molto ridotta).
Il principale costo associato al sistema di riconoscimento automatico sarà
solo l'investimento iniziale per l'acquisto del sistema stesso.